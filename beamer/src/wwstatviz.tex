\begin{frame}
  \frametitle{API}
% it's a python package presenting an API for simple generation of
% visualizatons
% what is an API

Available plots in the current state are:
\begin{itemize}
  \item choropleth: ...
  \item heatmap: ...
  \item line plot: ... for time series visualization
  \item histogram: ...
\end{itemize}
% It's in active development, we will extend it to more widgets in the future
\end{frame}

\begin{frame}[fragile,shrink=30]
  \frametitle{Package structure}
\begin{verbatim}
wwstatviz
|-- __init__.py
|-- generators/
|   |-- __init__.py
|   |-- choropleth.py
|   |-- generator.py
|   |-- heatmap.py
|   `-- line.py
|-- io/
|   |-- __init__.py
|   |-- csvreader.py
|   |-- jsonreader.py
|   |-- iso.py
|   |-- reader.py
|   `-- writer.py
|-- figure.py
`-- visualizer.py
\end{verbatim}
\end{frame}

\begin{frame}[fragile,shrink=10]
  \frametitle{The Visualizer Class}
  \framesubtitle{Input Data}
  
  The constructor of the class takes as input a data file \\
  (in the CSV format):
  \begin{itemize}
    \item The first line must contain the header
    \item Each row must start with a country code (ISO-3166 2-digit or 3-digit)
    \item The columns represent the features of the data
  \end{itemize}
  
  \vspace{5mm}
  
  Example:
  \begin{verbatim}
    ,f1,f2,f3
    AFG,0,1,2
    BEL,5,4,3
    FRA,6,7,8
    SEN,12,13,14
    USA,3,33,8
  \end{verbatim}

\end{frame}

\begin{frame}
  \frametitle{The Visualizer Class}
  \framesubtitle{Main functions}

  \begin{itemize}
    \item The Visualizer class is the main interface of the API (it orchastrates the differents tasks/actions).
    \item The constructor (\_\_init\_\_(self, data\_path)) takes as input 
      the path to the data file, and calls the corresponding reader from the ``io'' subpackage
    \item It contains functions (methods) for generating graphics (choropleth, histogram, etc.)
    \item These function are simple calls to the Generators
    \item Each function returns a Figure object (for later use, show()/save())
  \end{itemize}

\end{frame}

\begin{frame}[fragile,shrink=30]
  \frametitle{Generators}

  About Generators:
  \begin{itemize}
    \item Generators are responsible for producing plots.
    \item Each generator should inherit from the base class 
      ``Generator'' and must implement a ``generate()'' method
    \item The generator constructor takes as argument the different options 
      to be used for generating the plots 
      (e.g. whether or not to draw a legend, the countries to use, etc.)
  \end{itemize}

  \vspace{5mm}

  Example:
  \begin{lstlisting}[language=Python]
  class XYZGenerator(Generator): # class inheritance
      
      def __init__(self, ...): # this is the contructor
          ...

      # generate method must be implemented 
      # and must return a figure
      def generate(self): 
          ...
          return figure
  \end{lstlisting}
\end{frame}

\begin{frame}
  \frametitle{Input/Output module}
  About the ``io'' module:
  \begin{itemize}
    \item The ``io'' module is responsible for:
      \begin{itemize}
        \item reading data files from disk for different 
          formats (csv, json, etc.) % json is not implemented yet
        \item writing generatred figures to disk
      \end{itemize}
    \item It contains the base classes Reader and Writer
    \item For each data format, a reader submodule must be implemented
    \item Each reader submodule (e.g. csvreader) should implement a class that:
      \begin{itemize}
        \item inherits from the base class ``Reader''
        \item implements a ``read()'' method
      \end{itemize}
  \end{itemize}
\end{frame}
